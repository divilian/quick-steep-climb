
\chapter{Matrices}

It's now time for the granddaddy of all linear algebra entities: the
\textbf{matrix}. When we've finished this part of our climb, you'll actually be
able to see the summit we'll eventually reach.

By the way, the plural of \textit{matrix} is \textbf{matrices} (pronounced
MAY-trih-sees), kind of like the plural of \textit{index} is \textit{indices.}
But don't forget the singular is still ``\textit{matrix}!'' Don't let me (or
anyone else) catch you uttering the non-word ``matrice'' -- you'll sound like a
dweeb and drive me up a wall.

\section{Row and column vectors}
\index{vector}

Up to now, a vector has simply been a vector. I haven't made a big deal about
how you write it on the page. We've been free to write a vector
$\overrightarrow{\textbf{x}}$ with the three elements 6, 2, and 9 in either of
these ways:

\vspace{-.15in}
\begin{center}
\begin{tabular}{ccc}
$\overrightarrow{\textbf{x}}$ = \textbf{[}$\ 6\ \ 2\ \ 9\ $\textbf{]} &
\quad\quad \textit{...or...} \quad\quad &
$\overrightarrow{\textbf{x}}$ = $\begin{bmatrix} 6 \\ 2 \\ 9 \end{bmatrix}$ \ 
\end{tabular}
\end{center}
\vspace{-.15in}

\index{function}

Or heck, you could even write it diagonally if you want. This flexibility is
because all that really matters is the \textit{function} view of a vector that
we discussed in section~\ref{vectorIsFunction}. All that ultimately matters is
that you associate the correct index number with the correct element. However I
might draw $\overrightarrow{\textbf{x}}$ on paper, if I asked you for the value
of ``element \#0,'' you'd say 6, and if I asked for ``element \#2,'' you'd say
9. The way it looks has been immaterial up until now.

\index{row vector}
\index{column vector}

That will still be true sometimes. But beginning with this chapter, it's going
to sometimes turn out to matter whether or not we think of a vector as a
\textbf{row vector} (the left-hand-side version of
$\overrightarrow{\textbf{x}}$, above) or a \textbf{column vector} (the
right-hand-side). Memorize these terms: they matter, and you'll have to have
them on the tip of your neural cortex. A row goes horizontally, side-to-side;
and a column goes vertically, up-to-down.

I'll try to always be very careful to emphasize the row vs.~column nature of a
vector in those cases where it turns out to matter.

\index{default}

By the way, one surprising thing (at least, it was to me) is that the
``default'' is for an unspecified vector to be treated as a \textit{column}
vector, not a row. Column vectors take up more room on the page, and aren't as
natural when you're writing on paper, which I guess is why it surprised me. At
any rate, whenever a vector is under discussion, try to visualize it as an
up-and-down column of entries, unless the accompanying text explicitly says
otherwise.

\section{The matrix}

\index{matrix (plural: matrices)}

At last, the matrix. This will seem underwhelming at first, but \textit{boy}
does it pack a wallop.

A matrix is simply a two-dimensional rectangular grid of entries, kind of like
a spreadsheet. We'll use capital letters to designate them, with no special
arrow-like or other adornment. Here's our first example:

\vspace{-.15in}
\begin{align*}
A =
\begin{bmatrix}
5 &-7 &3 &9 \\
18 &4 &1 &1 \\
3 &-3 &\pi &4 \\
\end{bmatrix}
\end{align*}
\vspace{-.15in}

\index{dimension}

Matrices are always rectangular, but not always square. The $A$ matrix is
called a ``$3\times 4$'' (three-by-four) matrix, since it has three
\textbf{row}s and four \textbf{column}s. We say that $3\times 4$ are the
matrix's \textbf{dimensions}. Again, it's important to master all this
terminology. When giving the dimensions of a matrix, you always list the number
of rows first, and then the number of columns.

\index{index number}

To specify an individual element, we need \textit{two} indices instead of just
one as we did for a vector. We'll use Python-style numbering (starting with 0)
and write the row and column as a two-part comma-separated subscript:

\vspace{-.15in}
\begin{align*}
A_{0,0} &= 5 \\
A_{1,0} &= 18 \\
A_{0,3} &= 9 \\
A_{2,2} &= \pi \\
\end{align*}
\vspace{-.25in}

Just practice first moving down to the correct row, then moving over to the
correct column, and you'll be fine.

\subsection{Labels}

\index{label}

As with vectors, we won't always use index numbers to designate rows and
columns: sometimes we'll use labels. Check out this matrix $W$ (for
``weather''):

\vspace{-.4in} 
\begin{adjustwidth}{}{60pt}
\begin{center}
\begin{multicols}{2}
\begin{flushright}
\hspace*{1cm} \\
\hspace*{1cm} \\
\footnotesize{D.C.} \\
\footnotesize{Fredericksburg} \\
\footnotesize{Richmond} \\
\end{flushright}
\columnbreak
\vspace{-1.5in} 
\begin{align*}
\begin{bmatrix}
81 \ & \ 86 \ & \ 78 \ & \ 74 \ & \ 77 \\
83 \ & \ 86 \ & \ 79 \ & \ 79 \ & \ 82 \\
82 \ & \ 86 \ & \ 84 \ & \ 87 \ & \ 87 \\
\end{bmatrix}
\end{align*}
\vspace{-.15in}
\scriptsize{Mon} \ \  \scriptsize{Tue} \ \ \scriptsize{Wed} \ \ \scriptsize{Thu} \ \ \scriptsize{Fri} \\
\end{multicols}
\end{center}
\end{adjustwidth}
\vspace{-.15in}

Here we're using city names for the row labels, and days of the week as the
column labels. It's still easy peasy to interpret -- how hot did it get in the
nation's capital on Tuesday? 86\textdegree, of course. Using the same subscript
notation as above, we could say:

\vspace{-.15in}
\begin{align*}
W_{\textrm{D.C.},\textrm{Mon}} &= 81 \\
W_{\textrm{Fredericksburg},\textrm{Wed}} &= 79 \\
W_{\textrm{Fredericksburg},\textrm{Thu}} &= 79 \\
W_{\textrm{Richmond},\textrm{Thu}} &= 87 \\
&\vdots \\
\end{align*}

and so forth. D.C. and Fred had a bit of a cool-down midweek, thank God, while
Richmond was all the while cooking in the upper 80's.

\section{A matrix is also a function}

\index{function}
Remember back in section~\ref{vectorIsFunction} (p.~\pageref{vectorIsFunction})
when I explained that a vector, viewed in a sufficiently weird way, was
actually a function? The same thing is true for matrices, just by adding one
more input to the function.

Put another way, let's consider the row labels (or numbers, if we want to be
boring) as the set $C$ (for ``cities''). And let's consider the column labels
as the set $D$ (for ``days-of-the-week''). Then, you can see that a matrix is
precisely maps a pair of a city and a day to a high temperature. (The high
temperatures are in the set $\mathbb{R}$, which are the real numbers.) In
symbols, $W$ is defined as this function:

\vspace{-.15in}
\begin{align*}
W : C \times D \rightarrow \mathbb{R}
\end{align*}
\vspace{-.15in}

\index{domain}
\index{codomain}
\index{Cartesian product}
Recall that function syntax. $W$ is the name of the function. The part before
the arrow is the \textbf{domain} of the function: the set which its inputs are
drawn from. Since it's the Cartesian product of two sets (cities and days) this
domain is really all the ordered pairs of cities-and-days, like (D.C., Thurs)
and (Richmond, Monday). The function takes any ordered pair like that and gives
you a number telling you how hot that city was on that day. It's a snap when
seen this way.


\section{Matrix operations}

Just as section~\ref{vectorOps} listed the permissible actions we could perform
on vectors (and scalars), so this section lists the operations we can perform
on matrices (and vectors, and scalars). There's one other big one which I'll
save for entire separate chapter, but there are still four useful ones we'll
cover here.

\subsection{Operation \#1: scalar-matrix multiplication}

This one's a piece of cake. Recall that multiplying a scalar by a vector
amounted to multiplying the scalar by each of its elements, producing a vector
of the same dimension. Same here: we get a matrix of the same dimension by
multiplying individually:

\vspace{-.15in}
\begin{align*}
4 \cdot
\begin{bmatrix}
3 & 2 & 9 \\
1 & -1 & 0 \\
\end{bmatrix}
=
\begin{bmatrix}
12 & 8 & 36 \\
4 & -4 & 0 \\
\end{bmatrix}.
\end{align*}
\vspace{-.15in}

Sometimes we'll put a dot between the two, as above, though we'll often omit
that and just write the scalar and matrix side-by-side. Either way, it means
scalar-matrix multiplication.

\subsection{Operation \#2: matrix addition}

Also a piece of cake, and just what you'd expect:

\vspace{-.15in}
\begin{align*}
\begin{bmatrix}
4 & 1 \\
1 & -2 \\
3 & 18 \\
\end{bmatrix}
\begin{bmatrix}
1 & 2 \\
5 & 2 \\
-10 & -10 \\
\end{bmatrix}
+
\begin{bmatrix}
5 & 3 \\
6 & 0 \\
-7 & 8 \\
\end{bmatrix}.
\end{align*}
\vspace{-.15in}

The only hard part is not going cross-eyed as you zigzag your eyeballs across
the page to match up entries.

As with vector addition, you simply can't add two matrices at all if they don't
have the same dimensions. Also just like vectors, we can \textit{subtract} one
matrix from another just by adding the first matrix to ``$-1$ times the second
matrix.''

%transpose
%2x3 transpose gives us 3x2.  (Ai,j = ATj,i)
%
%use transpose on a row vector to get a column vector.
%(Treating the vector as a one-row / one-col matrix when we do this)
%
%
%matrix-vector multiplication. not what you'd expect. not even the kind of
%result you'd expect.
%
%Make A a 4x2, x a column vector 2x1.
%if mxn matrix, you need a n-dimensional column vector (or "nx1 matrix")
%weird! # of rows of one must be # of columns of the other.
%get an mx1 answer!
%
%two different ways to think about this A dot xvector.
%
%1. All the dot products of the vector with the matrix's rows.
%2. A linear combination of the matrix's columns (with the vector's elements as
%coefficients.)
%
%Show the second thing all worked out as the linear combo of columns.
%
%
%Why think about it as way 1? Jezebel col vector, matrix of all the guys (heck,
%let's put guys surveys in columns too, then that requires transpose
%
%
%Why think about it in way 2? Recipes matrix. each column is one recipe, each
%row is an ingredient. Vector shows how many of each recipe you want to make.
%
