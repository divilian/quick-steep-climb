
\chapter{Matrices}

It's now time for the granddaddy of all linear algebra entities: the
\textbf{matrix}. When we've finished this part of our climb, you'll actually be
able to see the summit we'll eventually reach.

By the way, the plural of \textit{matrix} is \textbf{matrices} (pronounced
MAY-trih-sees), kind of like the plural of \textit{index} is \textit{indices.}
But don't forget the singular is still ``\textit{matrix}!'' Don't let me (or
anyone else) catch you uttering the non-word ``matrice'' -- you'll sound like a
dweeb and drive me up a wall.

\section{Row and column vectors}
\index{vector}

Up to now, a vector has simply been a vector. I haven't made a big deal about
how you write it on the page. We've been free to write a vector
$\overrightarrow{\textbf{x}}$ with the three elements 6, 2, and 9 in either of
these ways:

\vspace{-.15in}
\begin{center}
\begin{tabular}{ccc}
$\overrightarrow{\textbf{x}}$ = \textbf{[}$\ 6\ \ 2\ \ 9\ $\textbf{]} &
\quad\quad \textit{...or...} \quad\quad &
$\overrightarrow{\textbf{x}}$ = $\begin{bmatrix} 6 \\ 2 \\ 9 \end{bmatrix}$ \ 
\end{tabular}
\end{center}
\vspace{-.15in}

\index{function}

Or heck, you could even write it diagonally if you want. This flexibility is
because all that really matters is the \textit{function} view of a vector that
we discussed in section~\ref{vectorIsFunction} (p.~\pageref{vectorIsFunction}).
No matter which way you write it, if I asked you for the value of ``element
\#0,'' you'd say 6, and if I asked for ``element \#2,'' you'd say 9. The way
it's drawn on paper has been immaterial.

\index{row vector}
\index{column vector}

That will still be true sometimes. But beginning with this chapter, it's going
to sometimes turn out to matter whether or not we think of a vector as a
\textbf{row vector} (the left-hand-side version of
$\overrightarrow{\textbf{x}}$, above) or a \textbf{column vector} (the
right-hand-side). Memorize these terms: they matter, and you'll have to have
them on the tip of your neural cortex. A row goes horizontally, side-to-side;
and a column goes vertically, up-to-down.

I'll try to always be very careful to emphasize the row vs.~column nature of a
vector in those cases where it turns out to matter.

\index{default}

By the way, one surprising thing (at least, it was to me) is that the
``default'' is for an unspecified vector to be treated as a \textit{column}
vector, not a row. Column vectors take up more room on the page, and aren't as
natural when you're writing on paper, which I guess is why it surprised me. At
any rate, whenever a vector is under discussion, try to visualize it as an
up-and-down column of entries, unless the accompanying text explicitly says
otherwise.

\section{The matrix}

\index{matrix (plural: matrices)}

At last, the matrix.

%A matrix, capital letter, with no arrow on the top
%
%(example)
%
%always rectangle, but not always square
%
%rows and columns, list as row x column.
%
%
%A_1,2 is element at row 1, column 2  (we'll use Python's numbering system)
%
%Could do it with labels instead
%
%    Mon Tue Wed Thu, and Fred, Rich, DC
%
%    Cartesian product. A matrix is a function from Cities x Days -> Reals
%


