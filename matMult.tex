
\subsection{Outer and inner products}

Because we can treat a vector as a sort of degenerate matrix (with only one
row, or only one column), it sometimes makes sense to do this matrix-vector
multiplication with two vectors. Which one is treated as a row vector and which
one is treated as a column vector makes all the difference.

As an illustration, I'm going to define vectors $\overrightarrow{\textbf{p}}$
and $\overrightarrow{\textbf{q}}$ this way:

\vspace{-.15in}
\begin{align*}
\overrightarrow{\textbf{p}} =
\begin{bmatrix}
3 & 1 & 2 \\
\end{bmatrix}, \quad 
\overrightarrow{\textbf{q}} =
\begin{bmatrix}
5 \\  4 \\ -3 \\
\end{bmatrix}.
\end{align*}
\vspace{-.15in}

So $\overrightarrow{\textbf{p}}$ is a row vector, and
$\overrightarrow{\textbf{q}}$ is a column vector.

Now, treating $\overrightarrow{\textbf{p}}$ as a $1\times 3$ matrix, we perform
multiplication and get:

\vspace{-.15in}
\begin{align*}
\overrightarrow{\textbf{p}} \cdot \overrightarrow{\textbf{q}} =
\begin{bmatrix}
3 & 1 & 2 \\
\end{bmatrix} \cdot
\begin{bmatrix}
5 \\ 4 \\ -3 \\
\end{bmatrix} = 13.
\end{align*}
\vspace{-.15in}

\index{inner product}
It's just the dot product, of course, calculated in the usual way. Here, for
reasons that will shortly become clear, we also call it the \textbf{inner
product} of the two vectors.

Now suppose I swap the order, and compute $\overrightarrow{\textbf{q}}$ times
$\overrightarrow{\textbf{p}}$ instead. What would I get? The answer will surely
surprise you:

\vspace{-.15in}
\begin{align*}
\overrightarrow{\textbf{q}} \cdot \overrightarrow{\textbf{p}} =
\begin{bmatrix}
5 \\ 4 \\ -3 \\
\end{bmatrix} \cdot
\begin{bmatrix}
3 & 1 & 2 \\
\end{bmatrix} =
\begin{bmatrix}
15 & 5 & 10 \\
12 & 4 & 8 \\
-9 & -3 & -6 \\
\end{bmatrix}.
\end{align*}
\vspace{-.15in}

Hooooooo...\textit{what?!} $\overrightarrow{\textbf{p}}$ times
$\overrightarrow{\textbf{q}}$ is the number 13, but
$\overrightarrow{\textbf{q}}$ times $\overrightarrow{\textbf{p}}$ is an entire
grid full of numbers?

Yes it is. Here's why. 
