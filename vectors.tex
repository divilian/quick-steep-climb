
\chapter{Vectors}

\index{vector}
As I stated on p.~\pageref{mathematicalObject}, every ``algebra'' is comprised
of a set of mathematical objects which you can combine with various operations.
In linear algebra, those building blocks are \textbf{vector}s and
\textbf{matrices} (singular: matrix). Buried within them are many mysteries.
We'll cover them in considerable detail in this chapter and the next.

\section{Vector vs.~scalar quantities}

\index{scalar}
\index{scale of measure}

The first thing we should do is perhaps distinguish between a vector quantity
and a \textbf{scalar quantity}, which probably had the spotlight in most of
your previous math classes. A scalar value is simply a \textit{single} number,
like 5, or -3.2, or $\pi$, or even $9 + 2i$ if you're into imaginary numbers.
The word \textit{scalar} is related to the word \textit{scale}, as in a ``scale
of measure.'' Think of stepping on a scale to weigh yourself in the morning:
your scale gives you back a single number (which you may or may not like; I
won't judge).

\index{one-dimensional quantity}
\index{dimension}

We'll call scalars \textbf{one-dimensional} values. That might seem odd, since
we haven't really talked about ``dimensions,'' yet. But think of the plain-old
\textbf{number line} you learned about back in elementary school. Zero's drawn
in the middle, positive numbers to the right and negative numbers to the left,
and the whole thing extends infinitely in just one direction\footnote{It might
seem like ``two directions,'' since the number line goes both to the left and
to the right. But since \index{left-ness} \index{right-neww} left-ness is the
exact opposite of right-ness, these are considered ``the same direction''; it's
just a matter of how far you go back or forth you go along that one straight
path.}.

% TODO: draw number line

\index{tuna fish}
\index{stock price}

Examples are so ubiquitous they're hardly worth mentioning. A person's weight
in the morning is a scalar. A company's stock price on a given day is a scalar.
So is the net \textit{movement} (up or down) of a stock's price from one day to
the next. So is a respondent's answer to the survey question ``on a 1-to-10
scale, how much do you enjoy tuna fish?'' You can think of countless others.

We of course often use variables to represent scalar quantities, and in this
book we'll put a variable in italics (like ``\textit{x}'' or
``\textit{price}'') to signify that its underlying value is a scalar quantity.

\subsection{A vector: a multi-dimensional quantity}
\index{vector}

Now a vector quantity is kind of the same thing, except that it represents
\textit{more than one} value. Suppose we wanted to represent not just a stock's
price on a given day, but an entire year's worth of prices on consecutive days.
Then, we would need a vector quantity. Instead of a survey respondent's answer
to just one question, we might want to store her entire set of answers to all
twenty questions on the survey. Instead of tracking just my weight, I might
want to record my weight, height, BMI (body-mass index), and pulse, all at a
given moment.

\index{losing information}
\index{information loss}

Vectors are \textbf{multi-dimensional} quantities. And they can't be
represented on a number line. Let's say my weight is 210 lbs.~and my height is
6'2", or 74 inches. (This is not theoretical.) I could of course draw a number
line and put a dot at 74 and another dot at 210, but this wouldn't fully
represent the fact that my weight was 210 and my height was 74. For one thing,
the numbers are on completely different scales. (There's that word ``scale''
again.) For another, it's not clear which is which -- is the right-most point
supposed to be my height, or my weight? Trying to squeeze a two-dimensional
quantity into a one-dimensional number line would \textbf{lose information.}
We need a representation scheme that can accommodate the complexity of our
quantity.

\index{Cartesian plane}
\index{coordinate plane}

For a two-dimensional quantity like weight-and-height, the obvious choice is
the two-dimensional Cartesian plane. I've drawn the vector with my height and
weight on the left side of Figure~\ref{fig:vector}. You'll see that there's an
\textit{arrow} from the origin to the point (210,74), rather than just a
circular dot at that point, as you might have expected. This is because
sometimes, it turns out, we want to treat a vector as ``a net movement in a
particular direction for a particular distance.''


\begin{figure}[ht]
\centering
\includegraphics[width=0.4\textwidth]{vector.png}
\includegraphics[width=0.58\textwidth]{vectors.png}
\caption{Left: a two-dimensional vector, depicted in a Cartesian plane. Right:
several copies of \textit{the same vector}, shown originating at various
points. They're considered ``the same'' vector because they all go in the same
direction and have the same length; the point they start at is irrelevant.}
\label{fig:vector}
\end{figure}

You can see this illustrated on the right-hand side figure, where I've drawn
several copies of \textit{the same vector}. This may seem weird, but in terms
of pictures, here's how you want to think of it: \textbf{a vector has a
direction and a length, but not a starting point}. In the case of Stephen's
biometrics, the vector's direction is east-by-northeast-ish (19.4\textdegree\
to be exact, if you wanted to work out the trig) and its length is 222.6
(Pythagorean Theorem), but it doesn't have an intrinsic ``starting point'';
it's just an arrow pointing this-a-way and yea-far, no matter where it's
anchored.

\index{mosquito}

Interestingly, the word \textit{vector} comes from a root meaning ``to carry.''
You may have heard people describe mosquitoes as being ``vectors'' for a
particular disease -- this means that they carry that disease. By thinking of a
vector as an arrow like in Figure~\ref{fig:vector}, the ``carry''
interpretation might start to make sense. A vector can represent a
transposition from one point to another. If I grew 74 inches taller and gained
210 more pounds, my point on the Cartesian plane would move in the direction
and the distance of that arrow.

\subsection{Don't visualize this}

Now that was all for \textit{two} dimensions. What about vectors with three, or
five, or even twenty dimensions? Well, for the three-dimensional case you can
indeed draw a 3-d figure with three axes, and plot three-dimensional points on
it. It turns out that most humans, though, are positively horrible at
interpreting such plots. And when you move beyond three dimensions, it's
utterly hopeless. (A friend of mine in fourth grade claimed he could visualize
four dimensions in his head, but I didn't believe him and still don't.)

But importantly, that doesn't mean we won't ever deal with higher-dimensional
vectors. In fact, vectors with lots and lots of dimensions will come up
constantly for us in this book -- believe it or not, we'll do an example where
the vectors have 50,000 dimensions! :-O

Don't worry; this won't make your head explode. In fact, it's a lot easier than
you might think to work with very high-dimensional vectors. Consider the
example I gave above about tracking a company's stock price every day for a
year. That's just a list of 365 numbers, all in a row. How hard is that to
imagine?

To work with vectors of more than three dimensions, you only have to do one
thing: give up trying to visualize them in a geometric space. As I'll describe
in the next section, it only occasionally makes sense to think about vectors
geometrically anyway; much of the time, we'll simply think of them as
quantities that have more than one component, unlike their simple brethren the
scalars.

\smallskip

Finally, the notation we'll use for vector variables. Instead of putting the
variable name in italics, we'll put it in bold-face with an arrow on the top of
it, like this: $\overrightarrow{\textbf{x}}$. The individual components of the
vector will be listed in boxies (square-brackets) like this: $[\ -2\ \ 5.9\ \
17\ \ -3\ ]$. So, if we define $\overrightarrow{\textbf{stephen}}$ as the
vector with my height and weight in it, we would write:

\vspace{-.15in}

\begin{align*}
\overrightarrow{\textbf{stephen}} = [\ 210\ \ 74 \ ].
\end{align*}

\section{Five ways to think about vectors}

In the mathematical writings you'll encounter, computer scientists use the word
``vector'' in a variety of ways. They're all ultimately compatible with each
other, but they can seem disorientingly different at first. Really, it's a
tribute to how powerful the vector concept is that people use them in so many
ways for so many different things.

I'm going to suggest that there are \textit{five} different ways to think about
a vector, and I'm going to arrange these ways on a continuum from ``concrete''
to ``abstract.'' This spectrum is depicted in Figure~\ref{fig:fiveWays}.

\begin{figure}[ht]
\centering
\includegraphics[width=1\textwidth]{fiveWays.pdf}
\caption{Five ways to think about a vector.}
\label{fig:fiveWays}
\end{figure}

Let's take it from the bottom.

\subsection{1. A sequence of two coordinates}

\index{coordinate}
\index{element}
This is the height-weight example, in which something like
$\overrightarrow{\textbf{stephen}}$ is an ordered pair that can easily be
visualized on a two-dimensional Cartesian plane. Because of this plotting
aspect, I'll often call the two parts of the vector \textbf{coordinates}, but
as we create vectors with more pieces I'll more often call them
\textbf{elements}. These terms mean the same thing -- they both refer to the
individual components of the vector.

\index{index number}

We will need a way to select each of the coordinates individually, and for that
we use \textbf{index numbers} (sometimes abbreviated to simply
``\textbf{index},'' the plural of which is ``indices.'') As you can see in
Figure~\ref{fig:fiveWays}, I've put two smaller numbers directly below the two
coordinates of the bottom vector, to indicate that we call them ``coordinate
\#0'' and ``coordinate \#1.'' We'll also use the phrase ``\textbf{index into
the vector},'' where ``index'' is a verb. If I take that bottom-most vector,
and index into it at coordinate 1, I get the (scalar) value 93.

You might wonder why I didn't write ``1'' and ``2'' instead of starting with
zero, and the answer has to do with the fact that we'll be using Python in this
book. Every programming language has a way of indexing into its vector-like
objects, and Python, Java, and C++ all begin indexing with the number 0. There
are actually some good reasons for this, which I won't get into. It's not
universally embraced, however; languages like R and Julia start their indexing
at 1. Go figure.

I put this figure at the extreme ``concrete'' end of the spectrum, because it's
so nuts-and-bolts and easy to visualize. As we ascend up the hierarchy,
things will get less and less visualizable.


\subsection{2. A sequence of more than two coordinates}

As I mentioned earlier in this chapter, having more than two coordinates in a
vector isn't really all that weird...you simply have to give up any hope of
visualizing it geometrically. But it's easy enough to do: a list of four
numbers -- 89, 93, 70, and 133 -- is the most natural thing in the world. One
could imagine finding the sum of the elements, the maximum element, the number
of negative elements, or other more exotic things.

Again, our indexing starts at 0, and this time goes up to 3. Note what this
implies: if we have a vector with four elements, there is no element \#4! This
is a common pitfall for newcomers to the subject and to languages like Python.
If I have a vector with $n$ coordinates, those coordinates are numbered 0 up to 
$n-1$, but \textit{not} up to $n$.

\subsection{3. A collection with non-numeric indices}

At this point of the hierarchy, I change my nomenclature from ``sequence'' to
``collection.'' That's because here, we don't \textit{number} the elements of
our vector anymore but instead \textit{name} them. Thus there isn't any
meaningful order to the elements anymore -- instead of ``IQ \#0,'' ``IQ \#1,''
and so forth, we have ``Jezebel's IQ,'' ``Filbert's IQ,'' and the like. Nothing
super weird here, but things may be starting to look less and less math-y to
you.

% TODO: talk about how diff languages represent this in diff ways; with R, you
% can name the elements of a vector, whereas in Python, you'll need a dict.

\subsection{4. A collection with non-numeric elements}

Heck, if the indices don't have to be numbers, why would the elements need to
be? And indeed we will often have cause to work with vectors like the one in
place 4 of the continuum, in which there's not a number in sight. This example
holds the favorite colors for each of our four friends, which are of course
non-numeric.

\subsection{5. Just a ``thing''}

Finally, you won't see this usage of vectors until you get to some more
advanced math, but I'd be doing you a disservice if I didn't point it out here.
I remember the first time I read some research in which the author was going on
and on about ``vectors,'' and I was dreadfully confused because none of his
``vectors'' seemed to have any elements in them! I was like, ``what do you mean
\textit{vectors}, dude? Did your word processor auto-correct a different
word?''

\index{vector space}
\index{algebra@``an algebra''}

This computer scientist was treating ``vectors'' as whole objects, not even
considering what their elements were (or whether they even \textit{had} any
elements). He was working with an abstract notion called a \textbf{vector
space} which we'll touch on much later; for now I'll just tell you that it's
closely related to the notion of an algebra that we discussed in
Chapter~\ref{ch:intro}. He was taking advantage of some of the elegant results
presented later in this book, which are guaranteed to hold for whatever
mathematical objects you care to define, as long as you obey certain ground
rules -- whether those objects have any ``elements'' to them or not. I mention
this mostly to anchor your future self in solid ground the first time you
inevitably come across the use of ``vector'' as a very un-list-like thing.
You'll remember reading this, say to yourself ``ah yes -- Stephen warned me
once that the extreme abstract end of the continuum works like this,'' and
proceed with confidence. I won't say anything more about it in this book.

